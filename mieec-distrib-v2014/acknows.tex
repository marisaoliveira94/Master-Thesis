\chapter*{Agradecimentos}
%\addcontentsline{toc}{chapter}{Agradecimentos}
Quero agradecer antes de tudo aos meus pais, Fernando e Conceição, não só porque me puserem neste mundo e me deram a melhor educação que me poderiam ter dado, mas também porque me permitiram vir estudar para a melhor Faculdade de Engenharia do país e seguir o melhor caminho possível, ainda que tal viesse impor algumas restrições e cuidados. Também lhes quero agradecer porque ao longo destes 5 anos conseguiram aguentar as minhas crises existências, os meus choros quando as coisas não corriam tão bem, e foi também graças a isso que consegui continuar a ter força e lutar pelo que sempre quis. Agradeço ainda a minha irmã Elisabete, porque por muito chata que seja sempre me ajudou não só a nível pessoal mas também quando era preciso corrigir este tipo de documentos importantes, aliás, ela será a primeira a ler este documento e a queixar-se de que tenho muitos erros e não sei colocar virgulas e acentos. Obrigada por ter paciência para corrigir estes documentos super compridos cheios de termos técnicos e projetos que ela não faz ideia para que servem. Para além dela, agradeço também ao meu cunhado Marco, porque em conjunto conseguiram colocar no mundo os meus dois sobrinhos queridos: o Dinis e a Sofia. Ambos foram sempre o meu refugio ao fim de semana, mesmo quando as coisas não corriam bem sabia que no fim de semana chegaria a casa e eles estariam sempre lá. A todo o resto da minha família enorme, avós, tios, tias, primos, primas, primos emprestados e primas emprestadas, um agradecimento muito especial porque sempre me suportaram nesta minha caminhada na faculdade tornando as coisas mais fáceis de suportar durante a semana. Nem que isso implicasse o deslocamento de toda a família numa camioneta de 30 pessoas de Braga ao Porto, ou então percorrer 8km a pé ao sol só porque é bonito. A vocês, família, muito obrigada!

Quero agradecer também ao meu orientador, ao Professor João Canas Ferreira por me ter acompanhado neste percurso, orientando-me da melhor maneira possível. Ao Luís Pessoa e ao Professor Henrique Salgado por me terem dado a oportunidade de desenvolver esta dissertação no grupo de ótica e eletrónica do INESC-TEC dando-me o melhor apoio durante o desenvolvimento do mesmo. Quero agradecer ainda ao professor José Carlos Alves, pois apesar de nada ter a ver com o projeto ajudou-me sempre que lhe solicitei ajuda.  A todos os que partilharam comigo o laboratório, fica aqui um agradecimento especial por me terem ajudado sempre. Ao Hugo, ao Erik e à Joana por terem tido imensa paciência e dado força para continuar a trabalhar mesmo quando o projeto não parecia andar para a frente. Para além disso, quero agradecer de forma especial à Joana por todas as dicas que me deu quer na escrita da tese, quer nas apresentações feitas ao grupo ao longo do semestre mas também por se apresentar sempre disponível para qualquer dúvida que me surgia. 

Às meninas que partilharam comigo casa este ano, à Verónica e à Ritinha, fica um agradecimento enorme por todos os momentos partilhados que certamente não vou esquecer, mas também por terem sido sempre pacientes com a minha ansiedade. E também à minha antiga colega de casa, àquela chata da Nani, porque na realidade sem ela nunca teria tido coragem para vir estudar para a melhor Faculdade de Engenharia! Obrigada por todo o apoio que me deste nestes anos, e todos os momentos que partilhamos juntas.

Àqueles com quem partilhei estes últimos 5 anos, ficam as recordações, impossíveis de passar para um papel... Quero lhes agradecer por todos os momentos, por todas as chatices e por todas as aventuras que passamos juntos. Podia estar aqui a enumerar os nomes deles, mas sei que para bom entendedor meia palavra basta e 2012 será sempre um bom número. Aos outros com quem também tive chatices e muitos momentos bons não fica um agradecimento, mas sim uma palavra de apreço.

A todos os outros com quem partilhei apontamentos, ideias, estudos e trabalhos ao longo destes anos fica também um agradecimento especial. Entre eles os meninos do laboratório I224 (especialmente ao Diogo, ao Miguel e ao Artur), porque todos os \textit{post-it} que me foram dando ao longo do semestre foram sem dúvida motivacionais. 

Quero aqui ainda agradecer a uma pessoa que nada tem a ver com este projeto e conheci por fatalidade do destino. No entanto passou os últimos dias a rever todos os capítulos da minha tese, ainda que sem obrigação nenhuma. Ao Cardoso, um obrigado muito especial por toda a paciência que teve comigo, e por toda a paciência que teve para rever estas 140 páginas.

Por fim, mas definitivamente não por último, quero agradecer a uma pessoa que passou as últimas 72 horas a perceber todo o trabalho desenvolvido durante um semestre inteiro e que nada tem a ver com o assunto. A uma pessoa que abdicou de passar muitos momentos em família, como o São João, para me poder dar apoio na reta final deste percurso. Essa mesma pessoa que viu ao vivo todas as minhas angústias, ansiedades e momentos menos bons e mesmo assim manteve a cabeça fria para me colocar em ordem. E quando não era ao vivo, era à distância de uma chamada... À mesma pessoa que põe sempre tudo de lado para que eu possa estar bem. À pessoa que não me deixou ir ao fundo nestes últimos quase 4 anos. Àquele que, por muito que escreva e escreva, nunca vou conseguir agradecer o suficiente por tudo o que fez por mim.  A essa pessoa ``... fica aqui um grande \textit{Espaço} no fim ... '' para que eu possa agradecer .... muito obrigada!

\vspace{10mm}
\flushleft{Marisa Oliveira}
