\chapter{Conclusões e Trabalho Futuro} \label{chap:concl}

Neste capítulo são apresentadas as considerações finais sobre o projeto desenvolvido tendo em conta a qualidade dos resultados obtidos e também as principais dificuldades encontradas. Por fim, são apresentadas algumas propostas de trabalho futuro que visa melhorar o já desenvolvido.

\section{Considerações Finais} \label{chap6:cons_final}

\subsection*{Qualidade dos resultados obtidos}

O projeto desenvolvido passou por duas principais fases, e por isso foram obtidos os resultados que são de seguida apresentados.

O objetivo principal da primeira parte do projeto é explorar as diferentes configurações das placas HDMI e conseguir obter resultados que venham a ser aplicados na segunda fase do mesmo. Como tal, os resultados obtidos nesta fase foram ao encontro dos objetivos definidos, uma vez que foi possível explorar as diferentes configurações das placas e entender como se procede às suas reconfigurações. Foram desenvolvidas e validadas diversas arquiteturas para diferentes configurações das placas HDMI que devido às suas características vieram facilitar a segunda fase do projeto.

Relativamente à transmissão de dados em série, que é a segunda fase do projeto desenvolvido, também se conclui que o seu desenvolvimento veio ao encontro dos objetivos definidos para essa mesma fase. A transmissão em série de dados foi obtida com sucesso através do desenvolvimento de uma arquitetura que, em conjunto com o módulo GTX, é capaz de enviar dados e recebê-los corretamente.

Numa perspectiva geral do projeto, os objetivos foram cumpridos: a transmissão em série de dados entre dois dispositivos HDMI foi conseguida, ainda que possa vir a ser melhorada. 

\subsection*{Principais dificuldades encontradas}

Durante o desenvolvimento do projeto foram encontradas várias dificuldades em diferentes fases do mesmo que passam de seguida a ser expostas.

%%parte do HDMI
Do ponto de vista da primeira fase do projeto, as principais dificuldade encontradas foram a familiarização com as placas HDMI: perceber claramente o funcionamento das suas diferentes configurações e ainda o processo que se toma para a sua reconfiguração. Apesar de existirem manuais sobre as placas, nem todos os detalhes do funcionamento das mesmas são bem explícitos. Relativamente ao processo de reconfiguração das placas, não foi esta a principal dificuldade encontrada nesta fase, mas foi algo que demorou a ser bem entendido e que deve ficar documentado para trabalho futuro. 


%%parte do GTX
%-> Informação pouco clara sobre o GTX
%-> Muito tempo perdido a perceber os modulos GTX
%-> Curva de aprendizagem de declive elevado.
Relativamente à fase do projeto cujo objetivo é a utilização dos transcetores disponibilizados na placa FPGA VC7203 para transmissão de dados em série, foram encontradas diversas dificuldades. Isto porque o módulo GTX disponibilizado pela \textit{Xilinx} é muito completo, mas ao mesmo tempo também muito complexo levando a que a curva de aprendizagem relativamente a estes transcetores não seja linear. Apesar de existir diversa documentação relativamente a este tópico, existem certas conclusões que só se obtêm através da experimentação dos transcetores em simulação, o que leva a muito tempo empregue para perceber as suas características e o seu funcionamento num geral.


%%geral

Numa perspectiva geral do projeto, as principais dificuldades encontradas relacionam-se com os tempos de simulação, síntese e implementação das arquiteturas desenvolvidas. Todas estas foram simuladas entre cada uma das etapas até a programação final da FPGA. Para além disso houve ainda a necessidade de, em algumas arquiteturas, recorrer-se ao IP da \textit{Xilinx} que permite analisar os sinais internos da FPGA o que aumentava consideravelmente todos estes tempos mencionados anteriormente.

%-> Tempos de simulação comportamental, pos sintese, pos implementação elevados.
%-> Tempos de sintese e implementação elevados


\section{Trabalho futuro}

Todo o trabalho realizado teve como motivação a sua inclusão no projeto \textit{iBrow} no sentido de testar os transcetores desenvolvidos pelo mesmo. Consequentemente, para que tal inclusão seja realizada é necessário fazer algumas melhorias do trabalho desenvolvido, que passam de seguida a ser enumeradas:
\begin{itemize}
	\item \textbf{Protocolo de comunicação mais robusto}: Como já mencionado, neste projeto, optou-se por abordar a serialização de uma forma direta sem criar um protocolo de comunicação robusto. Assim sendo, o recurso a este procedimento vem melhorar o processo de transmissão de dados em série colmatando algumas falhas.
	
	\item \textbf{Implementação de códigos detetores de erros:} Com a inserção do sinal em canais ruidosos é expectável que os sinais sejam alterados. O recurso à implementação de códigos detetores de erros vai permitir que tais alterações dos dados sejam detetadas evitando assim que dados errados sejam transmitidos para o final da cadeia de transmissão.
\end{itemize}

Outro aspecto que pode ser explorado, como trabalho futuro, é a transmissão dos dados de som em série, visto que já foi possível obter essa mesma transmissão de uma forma direta. 


%-> Tornar a parte de serialização mais robusta
%
%-> Implementar códigos detetores de erros
%
%
%-> Passar o som na série

%-> pARA SER incluido no projeto ibrow tem de levar algum melhoramentos 