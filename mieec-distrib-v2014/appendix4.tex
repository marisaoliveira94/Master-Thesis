{\tiny \chapter{Configurações dos interruptores das placas HDMI} \label{ap4:switches}}

Este anexo contempla as diversas configurações possíveis dos interruptores presentes nas placas HDMI para as diversas configurações das mesmas.

\section{Configuração por omissão} \label{subsubsec:HDMIconfigdefault_switches}

Quando as placas estão configuras de fábrica, relembra-se que as imagens transmitidas correspondem ao formato RGB de 30 bits (10 bits por cor). Como tal, a indicação que vem em \cite{R009} sobre as funções dos interruptores da placa HDMI recetora é muito pouca. Apenas é indicado que quando esta configuração está ativa os interruptores se devem encontrar tal como especifica a tabela \ref{table:HDMI_default_switches_RX} na página \pageref{table:HDMI_default_switches_RX}.

%\begin{table}[h!]
%	\centering
%	\begin{tabular}{|c|c|}
%		\hline
%		\textbf{Interruptor} & \textbf{Estado} \\ \hline
%		\textbf{S1-1}        & ON              \\ \hline
%		\textbf{S1-2}        & ON              \\ \hline
%		\textbf{S1-3}        & ON              \\ \hline
%		\textbf{S1-4}        & ON              \\ \hline
%		\textbf{S1-5}        & Não usado       \\ \hline
%		\textbf{S1-6}        & Não usado       \\ \hline
%		\textbf{S1-7}        & Não usado       \\ \hline
%		\textbf{S1-8}        & ON              \\ \hline
%	\end{tabular}
%	\caption{Configuração dos interruptores da placa HDMI RX configurada de fábrica, adaptada de \cite{R009}}
%	\label{table:HDMI_default_switches_RX}
%\end{table}


\begin{table}[h!]
	\centering

		\begin{tabular}{@{}ll@{}}
			\toprule
			\multicolumn{1}{c}{\textbf{Interruptor}} & \multicolumn{1}{c}{\textbf{Estado}} \\ \midrule
			\textbf{S1-1}                            & ON                                  \\
			\textbf{S1-2}                            & ON                                  \\
			\textbf{S1-3}                            & ON                                  \\
			\textbf{S1-4}                            & ON                                  \\
			\textbf{S1-5}                            & Não usado                           \\
			\textbf{S1-6}                            & Não usado                           \\
			\textbf{S1-7}                            & Não usado                           \\
			\textbf{S1-8}                            & ON                                  \\ \bottomrule
		\end{tabular}%
	
	\caption{Configuração dos interruptores da placa HDMI RX configurada por omissão, adaptada de \cite{R009}}
	\label{table:HDMI_default_switches_RX}
\end{table}

Relativamente à placa HDMI transmissora, é sabido que lhe chegam imagens no formato RGB de 10 bits, no entanto é possivel configurar o ADV7511 de tal forma que na sua saída o número de bits não seja limitado a 10. Para tal é necessário configurar os interruptores da forma que a tabela \ref{table:HDMI_default_switches_TX} indica.

\begin{table}[h!]
	\centering
	\begin{tabular}{|c|c|c|c|}
		\hline
		\textbf{Interruptor} & \multicolumn{3}{c|}{\textbf{Estado}} \\ \hline
		\textbf{S1-1}        & OFF        & ON         & ON         \\ \hline
		\textbf{S1-2}        & ON         & ON         & OFF        \\ \hline
		\textbf{S1-3}        & ON         & OFF        & ON        \\ \hline
		\textbf{S1-4}        & ON         & ON         & ON         \\ \hline
		\textbf{OUTPUT}      & 8 bits     & 10 bits    & 12 bits    \\ \hline
		\textbf{S1-5}        & \multicolumn{3}{c|}{OFF}             \\ \hline
		\textbf{S1-6}        & \multicolumn{3}{c|}{Não usado}       \\ \hline
		\textbf{S1-7}        & \multicolumn{3}{c|}{Não usado}       \\ \hline
		\textbf{S1-8}        & \multicolumn{3}{c|}{Não usado}       \\ \hline
	\end{tabular}
	\caption{Configuração dos interruptores da placa HDMI RX configurada de fábrica, adaptada de \cite{R009}}
	\label{table:HDMI_default_switches_TX}
\end{table}
\subsubsection{Suporte de um canal de imagem e áudio} \label {subsubsec:HDMIconfig+audio_switches}

Quando se configuram as placas HDMI de forma a obter-se o suporte de áudio, então o formato da imagem transmitida também não é limitado a RGB. Desta maneira, o tabela \ref{table:HDMI_1ch+audio_switches_RX} indica como se devem configurar os interruptores de forma a obter-se na saída do ADV7612 as diversas possibilidades relativamente ao formato da imagem.
\begin{table}[h!]
	\centering
	\begin{tabular}{|c|c|c|c|c|c|c|}
		\hline
		\textbf{Interruptor}             & \multicolumn{6}{c|}{\textbf{Estado}}                                                                                                                                                                           \\ \hline
		\textbf{S1-1}                    & ON                                                        & OFF                                                       & ON                                                        & ON     & OFF     & ON      \\ \hline
		\textbf{S1-2}                    & ON                                                        & ON                                                        & OFF                                                       & ON     & ON      & OFF     \\ \hline
		\textbf{S1-3}                    & ON                                                        & ON                                                        & ON                                                        & OFF    & OFF     & OFF     \\ \hline
		\textbf{S1-4}                    & ON                                                        & ON                                                        & ON                                                        & ON     & ON      & ON      \\ \hline
		\multirow{2}{*}{\textbf{OUTPUT}} & \begin{tabular}[c]{@{}c@{}}YCbCr\\   444/422\end{tabular} & \begin{tabular}[c]{@{}c@{}}YCbCr\\   444/422\end{tabular} & \begin{tabular}[c]{@{}c@{}}YCbCr\\   444/422\end{tabular} & RGB    & RGB     & RGB     \\ \cline{2-7} 
		& 8 bits                                                    & 10 bits                                                   & 12 bits                                                   & 8 bits & 10 bits & 12 bits \\ \hline
		\textbf{S1-5}                    & \multicolumn{6}{c|}{ON}                                                                                                                                                                                        \\ \hline
		\textbf{S1-6}                    & \multicolumn{6}{c|}{ON}                                                                                                                                                                                        \\ \hline
		\textbf{S1-7}                    & \multicolumn{6}{c|}{ON}                                                                                                                                                                                        \\ \hline
		\textbf{S1-8}                    & \multicolumn{6}{c|}{ON}                                                                                                                                                                                        \\ \hline
	\end{tabular}
	\caption{Configuração dos interruptores da placa HDMI RX configurada para um canal e suporte de áudio, adaptada de \cite{R014}}
	\label{table:HDMI_1ch+audio_switches_RX}
\end{table}

À semelhança da placa recetora para esta configuração, também é possivel configurar o ADV7511 para se obter na sua saída diversos formatos de imagem. A tabela \ref{table:HDMI_1ch+audio_switches_TX} apresentada na página \pageref{table:HDMI_1ch+audio_switches_TX} indica essas mesmas combinações. 
\begin{table}[h!]
	\centering
	\begin{tabular}{|c|c|c|c|c|c|c|c|c|c|}
		\hline
		\textbf{Interruptor}             & \multicolumn{9}{c|}{\textbf{Estado}}                                                                                                                                                                                                                                                                                                                                       \\ \hline
		\textbf{S1-1}                    & ON                                                    & OFF                                                   & ON                                                    & ON                                                    & OFF                                                   & ON                                                    & ON     & OFF     & ON      \\ \hline
		\textbf{S1-2}                    & ON                                                    & ON                                                    & OFF                                                   & ON                                                    & ON                                                    & OFF                                                   & ON     & ON      & OFF     \\ \hline
		\textbf{S1-3}                    & ON                                                    & ON                                                    & ON                                                    & OFF                                                   & OFF                                                   & OFF                                                   & ON     & ON      & ON      \\ \hline
		\textbf{S1-4}                    & ON                                                    & ON                                                    & ON                                                    & ON                                                    & ON                                                    & ON                                                    & OFF    & OFF     & OFF     \\ \hline
		\multirow{2}{*}{\textbf{OUTPUT}} & \begin{tabular}[c]{@{}c@{}}YCbCr\\   444\end{tabular} & \begin{tabular}[c]{@{}c@{}}YCbCr\\   444\end{tabular} & \begin{tabular}[c]{@{}c@{}}YCbCr\\   444\end{tabular} & \begin{tabular}[c]{@{}c@{}}YCbCr\\   422\end{tabular} & \begin{tabular}[c]{@{}c@{}}YCbCr\\   422\end{tabular} & \begin{tabular}[c]{@{}c@{}}YCbCr\\   422\end{tabular} & RGB    & RGB     & RGB     \\ \cline{2-10} 
		& 8 bits                                                & 10 bits                                               & 12 bits                                               & 8 bits                                                & 10 bits                                               & 12 bits                                               & 8 bits & 10 bits & 12 bits \\ \hline
		\textbf{S1-5}                    & \multicolumn{9}{c|}{OFF}                                                                                                                                                                                                                                                                                                                                                   \\ \hline
		\textbf{S1-6}                    & \multicolumn{9}{c|}{Não usado}                                                                                                                                                                                                                                                                                                                                             \\ \hline
		\textbf{S1-7}                    & \multicolumn{9}{c|}{Não usado}                                                                                                                                                                                                                                                                                                                                             \\ \hline
		\textbf{S1-8}                    & \multicolumn{9}{c|}{ON}                                                                                                                                                                                                                                                                                                                                                    \\ \hline
	\end{tabular}
	\caption{Configuração dos interruptores da placa HDMI TX configurada para um canal e suporte de áudio, adaptada de \cite{R014}}
	\label{table:HDMI_1ch+audio_switches_TX}
\end{table}

\subsubsection{Suporte de dois canais de imagem melhorado} \label{subsubsec:HDMIconfigMelhorado_switches}

Quando se reconfigura as placas para suportarem a versão de transmissão de dois canais melhorada, é necessário ter em conta que existe um canal (canal 0) que tem a possibilidade de transmitir imagens tanto no formato YCbCr como RGB, porém o canal 1 apenas o faz no formato RGB. Na tabela \ref{table:HDMI_2ch_melhoradp_RX} da página \pageref{table:HDMI_2ch_melhoradp_RX} são apresentadas as configurações dos interruptores que configuram o ADV7612 de forma a enviar diferentes formatos.

\begin{table}[h!]
	\centering
	
	\begin{tabular}{|c|c|c|c|c|}
		\hline
		\textbf{Interruptor}             & \multicolumn{4}{c|}{\textbf{Estado}}             \\ \hline
		\textbf{S1-1}                    & ON            & OFF           & ON     & OFF     \\ \hline
		\textbf{S1-2}                    & ON            & ON            & ON     & ON      \\ \hline
		\textbf{S1-3}                    & ON            & ON            & OFF    & OFF     \\ \hline
		\textbf{S1-4}                    & ON            & ON            & ON     & ON      \\ \hline
		\multirow{2}{*}{\textbf{OUTPUT}} & YCbCr 444/422 & YCbCr 444/422 & RGB    & RGB     \\ \cline{2-5} 
		& 8 bits        & 10 bits       & 8 bits & 10 bits \\ \hline
		\textbf{S1-5}                    & \multicolumn{4}{c|}{ON}                          \\ \hline
		\textbf{S1-6}                    & \multicolumn{4}{c|}{ON}                          \\ \hline
		\textbf{S1-7}                    & \multicolumn{4}{c|}{ON}                          \\ \hline
		\textbf{S1-8}                    & \multicolumn{4}{c|}{ON}                          \\ \hline
	\end{tabular}
	\caption{Configuração dos interruptores da placa HDMI RX configurada para dois canais melhorados, adaptada de \cite{R013}}
	\label{table:HDMI_2ch_melhoradp_RX}
\end{table}

Relativamente à placa HDMI transmissora, ambos os canais são capazes de suportar imagens em formato RGB ou YCbCr. A tabela \ref{table:HDMI_2ch_melhoradp_TX} da página \pageref{table:HDMI_2ch_melhoradp_TX} apresenta as combinações dos interruptores para se poder obter os diversos formatos na saída do ADV7511.
\begin{table}[h!]
	\centering
	\begin{tabular}{|c|c|c|c|c|c|c|}
		\hline
		\textbf{Interruptor}             & \multicolumn{6}{c|}{\textbf{Estado}}                             \\ \hline
		\textbf{S1-1}                    & ON        & OFF       & ON        & OFF       & ON     & OFF     \\ \hline
		\textbf{S1-2}                    & ON        & ON        & ON        & ON        & ON     & ON      \\ \hline
		\textbf{S1-3}                    & ON        & ON        & OFF       & OFF       & ON     & ON      \\ \hline
		\textbf{S1-4}                    & ON        & ON        & ON        & ON        & OFF    & OFF     \\ \hline
		\multirow{2}{*}{\textbf{OUTPUT}} & YCbCr 444 & YCbCr 444 & YCbCr 422 & YCbCr 422 & RGB    & RGB     \\ \cline{2-7} 
		& 8 bits    & 10 bits   & 8 bits    & 10 bits   & 8 bits & 10 bits \\ \hline
		\textbf{S1-5}                    & \multicolumn{6}{c|}{OFF}                                         \\ \hline
		\textbf{S1-6}                    & \multicolumn{6}{c|}{Não usado}                                   \\ \hline
		\textbf{S1-7}                    & \multicolumn{6}{c|}{Não usado}                                   \\ \hline
		\textbf{S1-8}                    & \multicolumn{6}{c|}{ON}                                          \\ \hline
	\end{tabular}
	\caption{Configuração dos interruptores da placa HDMI RX configurada para dois canais melhorados, adaptada de \cite{R013}}
	\label{table:HDMI_2ch_melhoradp_TX} 
\end{table}
