\chapter*{Resumo}
%\addcontentsline{toc}{chapter}{Resumo}

A sociedade atual depende cada vez mais dos serviços de comunicações, exigindo melhores ligações e mais rápidas, prevendo-se num futuro próximo a necessidade de ligações na ordem das centenas de Gb/s. O projeto \textit{iBrow} que está a ser desenvolvido por vários parceiros, incluindo o INES-TEC, vem propor uma nova exploração do espetro de frequências permitindo assim comunicações de alta velocidade. Este projeto passa por propor uma metodologia que permite a manufaturação de transcetores de baixo custo capazes de atingir grandes débitos de transmissão. 

A interface HDMI é cada vez mais usada em todos os tipos de ambientes: tanto empresariais como domésticos. Por esse motivo acaba por ser uma boa interface para testar os transcetores que estão a ser desenvolvidos. E por isso nesta dissertação é proposto um projeto que visa testar os transcetores, desenvolvendo e implementando uma arquitetura em FPGA capaz de suportar sinais provenientes de uma fonte HDMI, serializá-los e ainda enviá-los através das saídas de alta velocidade para que possam ser de seguida enviados através dos transcetores do projeto \textit{iBrow}. A arquitetura é capaz de suportar ainda o processo inverso, isto é, receber os dados provenientes dos transcetores do projeto iBrow através das entradas de alta velocidade existentes na FPGA a ser utilizada e voltá-los a enviar para o dispositivo final HDMI.

Este projeto é dividido em duas partes para poder cumprir os requisitos. Inicialmente são desenvolvidas arquiteturas que permitem a comunicação entre dois dispositivos HDMI recorrendo-se a duas placas HDMI e uma FPGA de série 7 para realizar tal transmissão. As arquiteturas desta primeira parte do projeto variam entre a transmissão de imagens em diferentes formatos e também de imagem e som.Numa segunda fase do projeto desenvolveu-se uma arquitetura capaz de serializar dados e enviá-los pelas saídas de alta velocidade disponíveis na FPGA que foi utilizada. Esses mesmo dados foram recebidos na FPGA e convertidos novamente para o formato em paralelo. Por fim o projeto junta as duas partes e transmite dados HDMI em série através das entradas e saídas de alta velocidade e recupera o sinal.

\chapter*{Abstract}
%\addcontentsline{toc}{chapter}{Abstract}

Now I have to write it in English
