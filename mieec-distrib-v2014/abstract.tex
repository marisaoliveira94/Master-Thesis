\chapter*{Resumo}
%\addcontentsline{toc}{chapter}{Resumo}

A sociedade atual depende cada vez mais dos serviços de comunicações, exigindo melhores ligações e mais rápidas, prevendo-se num futuro próximo a necessidade de ligações na ordem das centenas de \SI{}{\giga\bit\per\second}. O projeto \textit{iBrow} que está a ser desenvolvido por vários parceiros, incluindo o INES-TEC, vem propor uma nova exploração do espetro de frequências permitindo assim comunicações de alta velocidade. Este projeto passa por propor uma metodologia que permite a manufaturação de transcetores de baixo custo capazes de atingir grandes débitos de transmissão. A interface HDMI é cada vez mais usada em todos os tipos de ambientes: tanto empresariais como domésticos. Por esse motivo acaba por ser uma boa interface para testar os transcetores que estão a ser desenvolvidos. E por isso, nesta dissertação, é proposto um projeto cuja motivação passa por testar os mesmos. 

O trabalho realizado consiste no desenvolvimento e implementação de  uma arquitetura em FPGA capaz de suportar sinais provenientes de uma fonte HDMI, serializá-los e ainda enviá-los a alta velocidade. A arquitetura suporta ainda o processo inverso, isto é, recebe os dados em série em alta velocidade e envia-os de seguida para um dispositivo HDMI de destino. Para cumprir os requisitos propostos o projeto é dividido em duas partes.

Numa primeira fase são desenvolvidas arquiteturas que permitem a comunicação entre dois dispositivos recorrendo a duas placas HDMI e uma FPGA de série 7 para realizar tal transmissão. Estas variam entre a transmissão de imagem em diferentes formatos e também de imagem e som. Numa segunda fase do projeto desenvolve-se uma arquitetura capaz de serializar dados e enviá-los pelas saídas de alta velocidade disponíveis na FPGA utilizada. Esses mesmo dados são recebidos de volta na FPGA e convertidos para o seu formato original. Por fim, englobando as duas partes, conseguiu-se obter a transmissão em série de dados HDMI.

\chapter*{Abstract}
%\addcontentsline{toc}{chapter}{Abstract}

Now I have to write it in English
