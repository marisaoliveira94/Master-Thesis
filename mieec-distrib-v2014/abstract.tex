\chapter*{Resumo}
%\addcontentsline{toc}{chapter}{Resumo}

A sociedade atual depende cada vez mais dos serviços de comunicações, exigindo melhores ligações e mais rápidas, prevendo-se num futuro próximo a necessidade de ligações na ordem das centenas de \SI{}{\giga\bit\per\second}. O projeto \textit{iBrow} que está a ser desenvolvido por vários parceiros, incluindo o INES-TEC, vem propor uma nova exploração do espetro de frequências permitindo assim comunicações de alta velocidade. Este projeto passa por propor uma metodologia que permite a manufaturação de transcetores de baixo custo capazes de atingir grandes débitos de transmissão. A interface HDMI é cada vez mais usada em todos os tipos de ambientes: tanto empresariais como domésticos. Por esse motivo acaba por ser uma boa interface para testar os transcetores que estão a ser desenvolvidos. E por isso, nesta dissertação, é proposto um projeto cuja motivação passa por testar os mesmos. 

O trabalho realizado consiste no desenvolvimento e implementação de  uma arquitetura em FPGA capaz de suportar sinais provenientes de uma fonte HDMI, serializá-los e ainda enviá-los a alta velocidade. A arquitetura suporta ainda o processo inverso, isto é, recebe os dados em série em alta velocidade e envia-os de seguida para um dispositivo HDMI de destino. Para cumprir os requisitos propostos o projeto é dividido em duas partes.

Numa primeira fase são desenvolvidas arquiteturas que permitem a comunicação entre dois dispositivos recorrendo a duas placas HDMI e uma FPGA \textit{Xilinx 	Virtex} 7 para realizar tal transmissão. Estas variam entre a transmissão de imagem em diferentes formatos e também de imagem e som. Numa segunda fase do projeto desenvolve-se uma arquitetura capaz de serializar dados e enviá-los pelas saídas de alta velocidade disponíveis na FPGA utilizada. Esses mesmo dados são recebidos de volta na FPGA e convertidos para o seu formato original. Por fim, englobando as duas partes, conseguiu-se obter a transmissão em série de dados HDMI.

\chapter*{Abstract}
%\addcontentsline{toc}{chapter}{Abstract}


Our society is increasingly dependent on communication services, demaning better and faster connections, reaching in a near future the order of hundreds of \SI{}{\giga\bit\per\second}. The \textit{iBrow} project, which is being developed by several partners, including INESC-TEC, comes with a new proposal of the frequency spectrum exploitation, allowing high-speed communications. This project also proposes a methodology that allows the manufacturing of low budget transceivers capable of reaching big transmission rates. The HDMI interface is increasingly used in all kinds of environment: from enterprises until domestic use. For that reason it is a good interface to test the transceivers that are being developed. Therefore, in this dissertation is proposed a project whose motivation is to test the \textit{iBrow} transceivers.

The progress work is based on the development and implementation of a design on the FPGA that is able to support signals from an HDMI sink, serialized them and send it through a high-speed channel. This architecture allows the reverse process, in other words, it is able to receive the high-speed serial data and send it to and HDMI source. To accomplish the proposed goals, the project is divided in two parts.

The firts step proposes architectures that allow the communication between two devices using two HDMI boards and a Xilinx Virtex 7 FPGA to reach the transmission. These architectures differs from each other regarding the kind of the transmission: different formats of image, or support of image and sound.

In the second phase of the project, an architecture which is able to serialize data and send it through the FPGA high-speed outputs is developed. It also supports the reception of the serial data and do the reverse process, which is convert it to its original format.

Finally, one can reach the final result by joining the two phases of the project, which is obtain a serial communication between HDMI sink and source.


