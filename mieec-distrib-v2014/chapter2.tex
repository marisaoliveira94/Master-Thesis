\chapter{Revisão Bibliográfica} \label{chap:sota}

Neste capítulo é realizada uma revisão bibliográfica das interfaces áudio e vídeo existentes, em específico do HDMI, também sobre métodos de codificação/descodificação de sinais HDMI numa FPGA e ainda sobre ligações de alta velocidade em série e cuidados que se deve ter com as mesmas.

\section{Interfaces de transmissão de video/audio}

As interfaces de áudio e vídeo definem parâmetros físicos e interpretações dos sinais recebidos, segundo \cite{R004}. Para sinais digitais a interface acaba por definir não só a camada física mas também a camada de ligação de dados e principalmente a camada da aplicação. As características físicas do equipamento (elétrico ou ótico) incluem o número e o tipo de ligações necessárias, tensões, frequências, intensidade ótica e ainda o design físico dos conectores. Relativamente à camada de ligação de dados, esta define como os dados da aplicação serão encapsulados para que, por exemplo, possam ser sincronizados ou para fazer correções de erros. Por fim, a camada da aplicação define o formato do sinal de áudio e vídeo a ser transmitido, normalmente incorporando codecs não específicos. No entanto, por vezes esta camada acaba por não definir em concreto o tipo de formato de dados deixando em aberto tal parâmetro para que se possa transmitir dados no geral (é o caso do HDMI). No caso dos sinais analógicos, todas as funções que existem para os sinais digitais definidas em três camadas, são representadas num único sinal.

No caso da transmissão de sinais de áudio e vídeo digital existem várias interfaces que passam a ser analisadas, segundo \cite{R004}:
\bullet \textbf{\textit{Display Port}}: utiliza um conector do tipo DisplayPort e é o principal concorrente do HDMI. Esta interface define uma interconexão sem licenças que foi inicialmente desenhada para ser utilizada numa conexão entre o computador e o monitor do mesmo. O sinal de vídeo não é compatível com DVI ou HDMI, mas um conector DisplayPort pode fazer passar estes sinais.

\bullet \textbf{\textit{IEEE 1394 “FireWire}}: utiliza um conector do tipo FireWire ou i.LINK. Este protocolo de transferência de dados é principalmente utilizado em câmaras digitais, mas também em computadores e em transferências de sinal de áudio. Este tipo de interface é capaz de hospedar vários sinais no mesmo cabo entregando os dados nos devidos destinos.

Neste capítulo usa-se texto de um artigo apresentado na Conferência
XATA2006~\citep{kn:MVL06-xata}.

Nos últimos tempos têm surgido diversas soluções, apresentadas por
empresas do sector Auto\-ma\-ção de Sistemas para a disponibilização de
sistemas \scadadms{} na \textit{Web}.

Fusce risus mi, tristique eu, consectetuer id, auctor sed, elit. Donec
laoreet. Duis consectetuer interdum libero. Etiam eu orci. In eu
arcu. Fusce luctus diam eget lectus. Duis interdum lacus sed
ligula. Proin vestibulum felis eget lacus. Vivamus vestibulum, tellus
ut congue viverra, mauris lacus tempor turpis, eu congue nisi magna at
dolor. Ut molestie vehicula libero. Praesent in neque sed risus tempus
ornare. Donec hendrerit, erat eu semper aliquam, pede nulla dapibus
risus, ut pretium orci pede et neque.
Etiam eget tortor a metus convallis viverra. Quisque eget nisi sed
orci facilisis interdum. Aliquam non felis. 

\section{Secção Exemplo}\label{sec:dialecto}

\emph{Scalable Vector Graphics}\index{SVG}\index{XML!SVG} é uma
linguagem em formato XML que descreve gráficos de duas dimensões. 
Este formato padronizado pela W3C (\emph{World Wide Web Consortium})
é livre de patentes ou direitos de autor e está totalmente
documentado, à semelhança de outros W3C
standards~\citep{kn:svgdoc}.

Sendo uma linguagem XML, o \svg{} herda uma série de vantagens: a
possibilidade de transformar \svg{} usando técnicas como
XSLT\index{XML!XSLT}, de embeber \svg{} em qualquer documento
XML\index{XML} usando \textit{namespaces} ou até de  
estilizar \svg{} recorrendo a CSS\index{CSS} (\emph{Cascade Style Sheets}). 
De uma forma geral, pode dizer-se que \svg{}s interagem bem com as
actuais tecnologias ligadas ao XML e à Web, tal como referido
em~\citep{kn:svgibm,kn:svgw3c}.

Lorem ipsum dolor sit amet, consectetuer adipiscing elit. Donec a
eros. Phasellus non nulla non massa venenatis convallis. In
porta. Mauris quis magna. Proin mauris eros, aliquet id, eleifend
vitae, semper quis, erat. Aliquam id lectus non odio dignissim
blandit. Vestibulum porttitor arcu ut ligula. Nunc quis
erat. Curabitur ipsum tortor, ornare vitae, dapibus pretium, hendrerit
sed, urna. Vestibulum ante ipsum primis in faucibus orci luctus et
ultrices posuere cubilia Curae; Phasellus bibendum, nulla eget varius
aliquam, tortor nulla sollicitudin quam, vel vestibulum nisl magna at
sem. Aliquam velit sapien, ultrices viverra, tempus quis, ultrices at,
dui. Aliquam sit amet justo. Quisque tristique, metus eu iaculis
sagittis, urna leo bibendum diam, a ultricies sem diam a augue. Mauris
consectetuer, libero vel euismod tincidunt, nisi metus viverra ante,
quis pretium sapien odio nec risus. Nunc semper auctor
nulla\footnote{Exemplo de nota de rodapé.}. 

\subsection{Sub-secção Exemplo} \label{batik} 

Batik é um conjunto de bibliotecas baseadas em \textit{Java} que
permitem o uso de imagens \svg{} (visualização, geração ou
manipulação) em aplicações ou \textit{applets}~\citep{kn:batik}.  
O projecto Batik\index{Batik} destina-se a fornecer ao programador
alguns módulos que permitem desenvolver soluções especificas usando
\svg~\citep{kn:svgdoc}. 

Lorem ipsum dolor sit amet, consectetuer adipiscing elit. Nunc eu
nulla. Pellentesque vitae nibh ultrices quam iaculis
convallis. Aliquam purus eros, varius eget, volutpat sodales,
imperdiet nec, lacus. Curabitur in elit sed sem rutrum posuere. Class
aptent taciti sociosqu ad litora torquent per conubia nostra, per
inceptos himenaeos. Duis sem. Praesent ultricies odio vel
sapien. Integer faucibus malesuada libero. Cras semper, dolor id
ullamcorper varius, magna risus volutpat felis, id pellentesque nulla
ante at erat. Integer sodales. 

Quisque sit amet odio. In at risus sit amet turpis interdum
posuere. Maecenas iaculis vehicula sem. Ut leo arcu, malesuada vel,
imperdiet id, dignissim a, purus. Duis eleifend, lectus non venenatis
dignissim, risus libero imperdiet mi, nec gravida massa libero sed
mauris. Nullam lobortis libero non sapien. Integer convallis iaculis
erat. Morbi dictum. Ut ultrices pellentesque velit. Cras ac
ante. Etiam in neque tincidunt lacus gravida vehicula. Proin et nisi. 

Vivamus non nunc nec risus tempor varius. Quisque bibendum mi at
dolor. Aliquam consectetuer condimentum risus. Aliquam luctus pulvinar
sem. Duis aliquam, urna et vulputate tristique, dui elit aliquet nibh,
vel dignissim magna turpis id sapien. Duis commodo sem id
quam. Phasellus dolor. Class aptent taciti sociosqu ad litora torquent
per conubia nostra, per inceptos himenaeos. 

\subsection{Sub-secção Exemplo}

Loren ipsum dolor sit amet, consectetuer adipiscing elit. 
Praesent sit amet sem. Maecenas eleifend facilisis leo. Vestibulum et
mi. Aliquam posuere, ante non tristique consectetuer, dui elit
scelerisque augue, eu vehicula nibh nisi ac est. Suspendisse elementum
sodales felis. Nullam laoreet fermentum urna. 

Duis eget diam. In est justo, tristique in, lacinia vel, feugiat eget,
quam. Pellentesque habitant morbi tristique senectus et netus et
malesuada fames ac turpis egestas. Fusce feugiat, elit ac placerat
fermentum, augue nisl ultricies eros, id fringilla enim sapien eu
felis. Vestibulum ante ipsum primis in faucibus orci luctus et
ultrices posuere cubilia Curae; Sed dolor mi, porttitor quis,
condimentum sed, luctus in. 

\section{Resumo ou Conclusões}

Aliquam erat volutpat. Nunc pede ipsum, porttitor eu, bibendum non,
bibendum nec, nisl. Maecenas eget mauris. Nullam pulvinar. Curabitur
rutrum commodo est. Nam sapien pede, interdum eu, accumsan ultrices,
venenatis sit amet, tellus. Praesent ac ante bibendum enim varius
suscipit. Donec enim. Proin nisi. Quisque libero turpis, varius ut,
elementum vel, pulvinar sed, nunc. 
