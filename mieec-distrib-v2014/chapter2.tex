\chapter{Revisão Bibliográfica} \label{chap:sota}

Neste capítulo é realizada uma revisão bibliográfica das interfaces áudio e vídeo existentes, em específico do HDMI, também sobre métodos de codificação/descodificação de sinais HDMI numa FPGA e ainda sobre ligações de alta velocidade em série e cuidados que se deve ter com as mesmas.

\section{Interfaces de transmissão de video/audio}

As interfaces de áudio e vídeo definem parâmetros físicos e interpretações dos sinais recebidos, segundo \cite{R004}. Para sinais digitais a interface acaba por definir não só a camada física mas também a camada de ligação de dados e principalmente a camada da aplicação. As características físicas do equipamento (elétrico ou ótico) incluem o número e o tipo de ligações necessárias, tensões, frequências, intensidade ótica e ainda o design físico dos conectores. Relativamente à camada de ligação de dados, esta define como os dados da aplicação serão encapsulados para que, por exemplo, possam ser sincronizados ou para fazer correções de erros. Por fim, a camada da aplicação define o formato do sinal de áudio e vídeo a ser transmitido, normalmente incorporando codecs não específicos. No entanto, por vezes esta camada acaba por não definir em concreto o tipo de formato de dados deixando em aberto tal parâmetro para que se possa transmitir dados no geral (é o caso do HDMI). No caso dos sinais analógicos, todas as funções que existem para os sinais digitais definidas em três camadas, são representadas num único sinal.

No caso da transmissão de sinais de áudio e vídeo digital existem várias interfaces que passam a ser analisadas, segundo \cite{R004}:
\begin{description}
	\item[$\bullet$  \textbf{\textit{Display Port:}}] utiliza um conector do tipo DisplayPort e é o principal concorrente do HDMI. Esta interface define uma interconexão sem licenças que foi inicialmente desenhada para ser utilizada numa conexão entre o computador e o monitor do mesmo. O sinal de vídeo não é compatível com DVI ou HDMI, mas um conector DisplayPort pode fazer passar estes sinais.
	\item[$\bullet$   \textbf{\textit{ IEEE 1394 “FireWire":}}] utiliza um conector do tipo FireWire ou i.LINK. Este protocolo de transferência de dados é principalmente utilizado em câmaras digitais, mas também em computadores e em transferências de sinal de áudio. Este tipo de interface é capaz de hospedar vários sinais no mesmo cabo entregando os dados nos devidos destinos.
	\item[$\bullet$   \textbf{\textit{HDMI (High Definition Multimedia Interface):}}]utiliza um conector do tipo HDMI e é uma interface de transmissão de sinal áudio/vídeo comprimida para transmissão de sinal digital descomprimida. 
\end{description}

\section{HDMI (\textit{High Definition Multimedia Interface})}\label{sec:dialecto}
O HDMI é uma interface de áudio e vídeo de alta definição que transporta dados áudio no formato não comprimido. Suporta num único cabo qualquer formato de vídeo em diversas resoluções e desde 2004 tem vindo a sofrer algumas alterações que vêm melhorar o desempenho da interface. 

Esta interface está dividida em diversos canais de comunicação que implementam determinados protocolos, entre os quais se destacam as seguintes de \cite{R002}:
\subsection{DDC - \textit{Display Data Channel} } \label{batik} 
É um conjunto de protocolos utilizado nas comunicações digitais entre um dispositvo de origem e um dispositivo final que permite a comunicação entre ambos. Estes protocolos permitem que o ecrã comunique com o seu adaptador quais os modos que consegue suportar e também que o dispositivo que liga ao ecrã consiga ajustar alguns parâmetros, como por exemplo o contraste e a luminosidade. EDID (Extended display identification data) é a estrutura standard para este tipo de comunicações que define as capacidades do monitor e os modos gráficos suportados pelo mesmo.  Este protocolo é utilizado pela source da comunicação do HDMI para obter os dados necessários do dispoistivo sink, no sentido de perceber quais os modos suportados pelo mesmo. Este canal é também ativamente usado para HDCP (High-Bandwith Digital Content Protection) que será descrito mais a frente.

\subsection{TMDS - \textit{Transition-Minimized Differential Signaling} } \label{batik} 
É uma tecnologia utilizada para transmissão de dados em série de alta velocidade utilizado em comunicações digitais. O transmissor implementa um algoritmo que reduz as interferências eletromagnéticas nos cabos e permite ainda uma recuperação robusta de clock no recetor. 

Em específico na interface HDMI, este protocolo divide a informação a transmitir em 3 principais pacotes e intercala a sua transmissão : Período de transmissão de vídeo, período de transmissão de dados e período de controlo. No primeiro período (período de transmissão de vídeo) são transmitidas os pixeis do vídeo em linha. No segundo período (o período de transmissão de dados) são transmitidos os dados de vídeo e os dados auxiliares à transmissão dentro dos respectivos pacotes. O terceiro período ocorre entre os dois anteriores. 

Para além de ser utilizada no HDMI, esta técnica é também utilizada em interfaces DVI.


\section{Resumo ou Conclusões}

Aliquam erat volutpat. Nunc pede ipsum, porttitor eu, bibendum non,
bibendum nec, nisl. Maecenas eget mauris. Nullam pulvinar. Curabitur
rutrum commodo est. Nam sapien pede, interdum eu, accumsan ultrices,
venenatis sit amet, tellus. Praesent ac ante bibendum enim varius
suscipit. Donec enim. Proin nisi. Quisque libero turpis, varius ut,
elementum vel, pulvinar sed, nunc. 
